\documentclass[twocolumn,landscape,10pt]{article}
\usepackage[thinc]{esdiff} % for typesettign derivatives
\usepackage{amsthm} % provides an enhanced version of LaTex's \newtheorem command
\usepackage{mdframed} % framed environments that can split at page boundaries
\usepackage{enumitem} % bulletin points or other means of listing things
\usepackage{amssymb} % for AMS symbols
\usepackage{amsmath} % so as to use align
\usepackage{latexsym} % so as to use symbols like \leadsto
\usepackage{mathrsfs} % for using mathscr for char like operators
\usepackage{commath} % for using norm symbol
\usepackage{mathtools} % for using environments like dcases
\usepackage{authblk} % for writing affiliations
\usepackage{graphicx} % for importing images
\graphicspath{{./images/}} % for the path to images, also always put label behind captions
\usepackage{textcomp} % for using degree symbol
\usepackage{hyperref} % for clickable link in the pdf & customizable reference text
\usepackage[all]{hypcap} % for clickable link to images instead of caption
\usepackage[margin=0.9in]{geometry} % default is 1.5in
% \usepackage[left=0.4in, right=0.4in, top=0.8in, bottom=0.8in]{geometry}
\usepackage[title]{appendix} % for attaching appendix
\allowdisplaybreaks % allow page breaking in display maths, like align
\usepackage{xcolor} % for setting color of a block of text, use \textcolor{<color>}{}
\usepackage[normalem]{ulem} % for strikethrough text, use \sout{}
% allow for more advanced table layout
\usepackage{booktabs}
\usepackage{multirow}
\usepackage{siunitx}
% for adjusting caption settings
\usepackage{caption}
\captionsetup[table]{skip=10pt}

\theoremstyle{definition}
\mdfdefinestyle{defEnv}{%
  hidealllines=false,
  nobreak=true,
  innertopmargin=-1ex,
}

% The following is for writing block of code
\usepackage{listings}
\usepackage{color}

\definecolor{dkgreen}{rgb}{0,0.6,0}
\definecolor{gray}{rgb}{0.5,0.5,0.5}
\definecolor{mauve}{rgb}{0.58,0,0.82}

% setting of the thickness of the 4 lines of box
\setlength{\fboxrule}{2pt}

% Use the following to change code language and related settings
\lstset{frame=tb,
  language=Python,
  aboveskip=3mm,
  belowskip=3mm,
  showstringspaces=false,
  columns=flexible,
  basicstyle={\small\ttfamily},
  numbers=none,
  numberstyle=\tiny\color{gray},
  keywordstyle=\color{blue},
  commentstyle=\color{dkgreen},
  stringstyle=\color{mauve},
  breaklines=true,
  breakatwhitespace=true,
  tabsize=3,
  literate={~} {$\sim$}{1}
}

\pagestyle{headings}
\author{Lectured by Panos Parpas}
\title{Computational Finance}
\affil{Typed by Aris Zhu Yi Qing}
\begin{document}
\maketitle
\tableofcontents

\newpage

\section{Introduction}

\begin{itemize}
    \item \textbf{\underline{Forward contract}}: a customized \emph{binding} 
        contract to buy/sell an asset at a specific price 
        (\textbf{\underline{forward price}}) on a \emph{future} date.
    \item \textbf{\underline{American option}}: gives the holder the
        \emph{right} to exercise the option at any time before the
        date of expiration.
    \item \textbf{\underline{European option}}: can only be exercised
        at the date of expiration.
    \item \textbf{\underline{call/put option}}: gives the holder the
        \emph{right} to buy/sell.
        \textbf{\underline{}}
    \item \textbf{\underline{Derivative asset}}: assets defined in terms of
        underlying financial asset.
\end{itemize} 

\section{The Binomial Model}

\subsection{The One Period Model}

\subsubsection{Definitions}

\begin{itemize}
    \item 
        \textbf{\underline{Bond}} price process is deterministic and given by
        \begin{align*}
            B_0 & = 1, \\
            B_1 & = 1 + R.
        \end{align*}
        where constant $R$ is the spot rate for the period.
    \item
        \textbf{\underline{Stock}} price process is stochastic and given by
        \begin{align*}
            S_0 & = s, \\
            S_1 & = s\cdot Z
        \end{align*}
        where $Z$ is a stochastic variable defined as
        \[
            Z =
            \begin{cases}
                u, & \text{with probability $p_u$}, \\
                d, & \text{with probability $p_d$}.
            \end{cases}
        \]
        with $p_u+p_d=1$ and the assumption that $d<u$.
    \item
        \textbf{\underline{Portfolio}} on the $(B, S)$ market is a vector
        \[
            h = (x, y).
        \]
        with a deterministic market value at $t=0$ and stochastic value at
        $t=1$.
    \item \textbf{\underline{Value Process}} of the portfolio $h$ is defined as
        \[
            V_t^h = xB_t+yS_t,\quad t=0,1,
        \]
        or, in more detail,
        \begin{align*}
            V_0^h & = x+ys, \\
            V_1^h & = x(1+R)+ysZ.
        \end{align*}
    \item \textbf{\underline{Arbitrage}} portfolio is an $h$ with the properties
        \begin{align*}
            V_0^h & = 0, \\
            V_1^h & > 0, \quad\text{\emph{with probability 1}}
        \end{align*}
\end{itemize}

\subsubsection{Contingent Claim Pricing}

\newmdtheoremenv[style=defEnv]{theorem}{Proposition}
\begin{theorem}
    The model $h$ is free of arbitrage $\iff$
    $d\le(1+R)\le u$.
\end{theorem}

\paragraph{Comments} The above proposition implies that $(1+R)$ is a convex
combination of $u$ and $d$, i.e.\
\[
    1+R=q_u\cdot u + q_d\cdot d,
\]
where $q_u$ and $q_d$ can be interpreted as probabilities for a new probability
measure $Q$ with $P(Z=u)=q_u$ and $P(Z=d)=q_d$. Denoting expectation w.r.t. this
measure by $E^Q$ and with the following calculation
\[
    \frac{1}{1+R}E^Q[S_1]=\frac{1}{1+R}[q_usu+q_dsd]=\frac{1}{1+R}\cdot
    s(1+R)=s,
\]
we arrive at the following relation
\[
    s=\frac{1}{1+R}E^Q[S_1].
\]

\newmdtheoremenv[style=defEnv]{martingale measure}[theorem]{Definition}
\begin{martingale measure}
    A probability measure $Q$ is \textbf{\underline{martingale}} 
    if the following condition holds:
    \[
        S_0=\frac{1}{1+R}E^Q[S_1].
    \]
\end{martingale measure}

\newmdtheoremenv[style=defEnv]{arbitrage free 1}[theorem]{Proposition}
\begin{arbitrage free 1}
    Arbitrage-free model $\iff$ $\exists$ martingale measure $Q$.
\end{arbitrage free 1}

\newmdtheoremenv[style=defEnv]{binomial model martingale}[theorem]{Proposition}
\begin{binomial model martingale}
    For the binomial model above, the martingale probabilities are
    \[
        \begin{cases}
            q_u=\dfrac{(1+R)-d}{u-d}, \\[2ex]
            q_d=\dfrac{u-(1+R)}{u-d}.
        \end{cases}
    \]
\end{binomial model martingale}

\begin{proof}
Solve the system of equation
\[
    \begin{cases}
        1+R=q_u\cdot u+q_d\cdot d\\
        q_u+q_d=1.
    \end{cases}
\]
\end{proof}

\newmdtheoremenv[style=defEnv]{contingent claim}[theorem]{Definition}
\begin{contingent claim}
    A \textbf{\underline{contingent claim}} (financial derivative) is any
    stchastic variable $X$ of the form $X=\Phi(Z)$, where $Z$ is the stochastic
    variable driving the stock price process above, $\Phi$ is the
    \textbf{\underline{contract function}}.
\end{contingent claim}

\newtheorem{contingent claim eg}[theorem]{Example}
\begin{contingent claim eg}
European call option on the stock with
strike price $K$. Assuming that $sd<K<su$, we have
\[
    X=
    \begin{cases}
        su-K, &\text{if $Z=u$},\\
        0, &\text{if $Z=d$},
    \end{cases}
\]
so the contract function is given by
\begin{align*}
    \Phi(u) & = su-K,\\
    \Phi(d) & = 0.
\end{align*}
\end{contingent claim eg}

\newmdtheoremenv[style=defEnv]{replicated and reachable}[theorem]{Definition}
\begin{replicated and reachable}
    A given contingent claim $X$ can be \textbf{\underline{replicated}} / is
    \textbf{\underline{reachable}} if
    \[
        \exists h \text{ s.t. } V_1^h=X,\;\text{with probability 1}.
    \]
    We call such an $h$ a
    \textbf{\underline{hedging}}/\textbf{\underline{replicating}} portfolio.
    If all claims can be replicated, then the market is
    \textbf{\underline{complete}}.
\end{replicated and reachable}

\newmdtheoremenv[style=defEnv]{X pricing based on V}[theorem]{Proposition}
\begin{X pricing based on V}
    If a claim $X$ is reachable with replicating portfolio $h$, then the only
    reasonable price process for $X$ is
    \[
        \Pi_t[X]=V_t^h, \;t=0,1.
    \]
    Here, ``reasonable'' means that $\Pi_0[X]\neq V_0^h\Rightarrow$ arbitrage
    possibility.
\end{X pricing based on V}

\newmdtheoremenv[style=defEnv]{no arbitrage leads to complete}[theorem]{Proposition}
\begin{no arbitrage leads to complete}
    The general binomial model is free of arbitrage $\Rightarrow$ it is
    complete.
\end{no arbitrage leads to complete}

\begin{proof}
    Say a claim $X$ has contract function $\Phi$ s.t. \[
        V_1^h=
        \begin{cases}
            \Phi(u), & \text{ if $Z=u$}\\
            \Phi(d), & \text{ if $Z=d$},
        \end{cases}
    \]
    we can obtain the following system of equations by expanding $V_1^h$
    \begin{align*}
        (1+R)x+xuy&=\Phi(u),\\
        (1+R)x+xdy&=\Phi(d),
    \end{align*}
    and solve it to find out the replicating portfolio as
    \begin{align*}
        x&=\frac{1}{1+R}\cdot \frac{u\Phi(d)-d\Phi(u)}{u-d},\\
        y&=\frac{1}{s}\cdot \frac{\Phi(u)-\Phi(d)}{u-d}.
    \end{align*}
\end{proof}

\newmdtheoremenv[style=defEnv]{claim pricing prop}[theorem]{Proposition}
\begin{claim pricing prop}
    If the binomial model is free of arbitrage, then the arbitrage free price of
    a contingent claim $X$ is
    \[
        \Pi_0[X]=\frac{1}{1+R}E^Q[X],
    \]
    where the martingale measure $Q$ is uniquely determined by the relation
    \[
        S_0=\frac{1}{1+R}E^Q[S_1].
    \]
\end{claim pricing prop}

\begin{proof}
    Using the results derived previously, we can find out that
    \begin{align*}
        \Pi_0[X]
        &= x + sy \\
        &= \frac{1}{1+R}\left[\frac{(1+R)-d}{u-d}\cdot
        \Phi(u)+\frac{u-(1+R)}{u-d}\cdot \Phi(d)\right]\\
        &= \frac{1}{1+R}\left[\Phi(u)q_u+\Phi(d)q_d\right]\\
        &= \frac{1}{1+R}E^Q[X].
    \end{align*}
\end{proof}

\subsection{The Multiperiod Model}

\subsubsection{Definitions}

\begin{itemize}
    \item Model
        \begin{itemize}
            \item The bond price dynamics is given by
                \[
                    B_{n+1}=(1+R)B_n,\quad B_0=1.
                \]
            \item The stock price dynamics is given by
                \[
                    S_{n+1}=S_n\cdot Z_n,\quad S_0=s.
                \]
                where $Z_0,Z_1,\ldots,Z_{T-1}$ are assumed to be i.i.d.
                stochastic variables, with only two variables $u$ and $d$ and
                probabilities
                \[
                    P(Z_n=u)=p_u,\quad P(Z_n=d)=p_d.
                \]
            \item Illustrating the stock dynamics by means of a tree, we can see
                that it is \textbf{\underline{recombining}}.
        \end{itemize}
    \item A \textbf{\underline{portfolio strategy}} is a stochastic process
        \[
            \left\{h_t=(x_t,y_t);\;t=1,\ldots,T\right\}
        \]
        s.t. $h_t$ is a function of $S_0,S_1,\ldots,S_{t-1}$. By convention,
        $h_0=h_1$.
        \begin{itemize}
            \item Interpretation of $h_t$: at $t-1$,
                $x_t$ and $y_t$ of bonds and stocks are bought and held until time $t$.
        \end{itemize}
    \item The \textbf{\underline{value process}} corresponding to the portfolio
        $h$ is defined by
        \[
            V_t^h=x_t(1+R)+y_tS_t.
        \]
    \item A portfolio strategy $h_t$ is said to be \textbf{\underline{self-financing}} 
        if $\forall t=0,\ldots,T-1$,
        \[
            x_t(1+R)+y_tS_t = x_{t+1}+y_{t+1}S_t
        \]
        which says that the market value of the ``old'' portfolio $h_t$ 
        equals to the purchase value of the ``new'' portfolio $h_{t+1}$.
    \item An \textbf{\underline{arbitrage}} possibility is a self-financing
        portfolio $h$ with the properties
        \[
            \begin{align*}
                V_0^h &= 0, \\
                P(V_T^h\ge{}0) &= 1, \\
                P(V_T^h>0) &> 0.
            \end{align*}
        \]
    \item The \textbf{\underline{martingale}} probabilities $q_u$ and $q_d$ are
        defined as the probabilities for which the following relation holds
        \[
            s=\frac{1}{1+R}E^Q[S_{t+1}|S_t=s]
        \]
\end{itemize}

\subsubsection{Binomial Model Pricing Algorithm}

\newmdtheoremenv[style=defEnv]{arbitrage cond}[theorem]{Proposition}
\begin{arbitrage cond}
    The model is free of arbitrage $\Rightarrow$ $d\le(1+R)\le u$.
\end{arbitrage cond}

\newmdtheoremenv[style=defEnv]{assumption on u and d}[theorem]{Assumption}
\begin{assumption on u and d}
    From now, assume that $d < u$ and $d\le(1+R)\le u$.
\end{assumption on u and d}

\newmdtheoremenv[style=defEnv]{contingent claim in multi-period model}[theorem]{Definition}
\begin{contingent claim in multi-period model}
    A \textbf{\underline{contingent claim}} is a stochastic variable $X$ of the
    form
    \[
        X=\Phi(S_T),
    \]
    where the \textbf{\underline{contract function}} $\Phi$ is some given real
    valued function.
\end{contingent claim in multi-period model}


\newmdtheoremenv[style=defEnv]{replicated and reachable in multi-period model}[theorem]{Definition}
\begin{replicated and reachable in multi-period model}
    A given contingent claim $X$ can be \textbf{\underline{replicated}} / is
    \textbf{\underline{reachable}} if
    \[
        \exists \text{ self-financing } h \text{ s.t. } V_T^h=X,\;\text{with probability 1}.
    \]
    We call such an $h$ a
    \textbf{\underline{hedging}}/\textbf{\underline{replicating}} portfolio.
    If all claims can be replicated, then the market is
    \textbf{\underline{(dynamically) complete}}.
\end{replicated and reachable in multi-period model}

\newmdtheoremenv[style=defEnv]{The binomial algorithm}[theorem]{(Binomial
Algorithm) Proposition}
\begin{The binomial algorithm}
    Consider a $T$-claim $X=\Phi(S_T)$, which could be replicated with a
    self-financing portfolio $h$. Let $k$ be the number of up-moves occurred. So
    $V_t(k)$ can be computed recursively as
    \[
        \begin{cases}
            V_t(k) &=
            \dfrac{1}{1+R}\left[q_uV_{t+1}(k+1)+q_dV_{t+1}(k)\right],\\[2ex]
            V_T(k) &= \Phi(su^kd^{T-k}).
        \end{cases}
    \]
    where the martingale probabilities $q_u$ and $q_d$ are
    \[
        \begin{cases}
            q_u &= \dfrac{(1+R)-d}{u-d}\\[2ex]
            q_d &= \dfrac{u-(1+R)}{u-d}.
        \end{cases}
    \]
    and the hedging portfolio is given by
    \[
        \begin{cases}
            x_t(k) &= \dfrac{1}{1+R}\cdot \dfrac{uV_t(k)-dV_t(k+1)}{u-d},\\[2ex]
            y_t(k) &= \dfrac{1}{S_{t-1}}\cdot \dfrac{V_t(k+1)-V_t(k)}{u-d}.
        \end{cases}
    \]
    \textbf{\emph{N.B.}} 
    In Fig 2.9 in notes, (the figure might be confusing) $(-22.5,5/8)$ is both $h_0$ and $h_1$,
    $(-42.5,95/120)$ is $h_2(1)$, $(-2.5,1/8)$ is $h_2(0)$, etc.
\end{The binomial algorithm}

\newmdtheoremenv[style=defEnv]{t claim arbitrage free price}[theorem]{Proposition}
\begin{t claim arbitrage free price}
    The arbitrage free price at $t=0$ of a $T$-claim $X$ is given by
    \[
        \Pi_0[X]=\frac{1}{{(1+R)}^{T}}\cdot E^Q[X],
    \]
    where $Q$ denotes the martingale measure, or more explicitly,
    \[
        \Pi_0[X]=\frac{1}{{(1+R)}^{T}}\cdot\sum_{k=0}^{T}
        \begin{pmatrix}
            T \\
            k
        \end{pmatrix} 
        q_u^kq_d^{T-k}\Phi(su^kd^{T-k}).
    \]
\end{t claim arbitrage free price}

\newmdtheoremenv[style=defEnv]{iff condition for arbitrage free}[theorem]{Proposition}
\begin{iff condition for arbitrage free}
    $d<(1+R)<u\iff$ free of arbitrage.
\end{iff condition for arbitrage free}


\section{Brownian Motion}

\subsection{Definitions}

\begin{itemize}
    \item Let $A, B$ be two random variables, we define \textbf{\underline{equal
        in distribution}} as
        \[
            A\stackrel{d}{=}B
            \iff
            P(A\in C)=P(B\in C)\;\forall \text{ possible sets $C$}.
        \]
    \item
        We say a random variable $X$ has \textbf{\underline{stationary
        increments}} if
        \[
            X_t-X_s\stackrel{d}{=}X_{t+h}-X_{s+h} \quad\forall h>0.
        \]
    \item 
        A stochastic process $B_t$ is called a 
        \textbf{\underline{Brownian motion}} if
        \begin{itemize}
            \item $B_0 = 0$,
            \item $B_t-B_s\sim N(0,t-s) \;\forall t\ge s\ge 0$, implying
                \emph{stationary} increments,
            \item it has \emph{independent} increments.
        \end{itemize}
        with
        \begin{itemize}
            \item $\mathbb{E}[B_t] = 0$,
            \item Cov$(B_t,B_s)$ = min$\left\{t,s\right\}$. Show this using
                (Practice!)
                \begin{itemize}
                    \item $\text{Cov}(B_t,B_s)=\mathbb{E}[(B_t-\mathbb{E}[B_t])(B_s-\mathbb{E}[B_s])]$
                    \item
                        $\text{Cov}(B_t,B_s)=\mathbb{E}[B_tB_s]-\mathbb{E}[B_t]\mathbb{E}[B_s]$.
                \end{itemize}
        \end{itemize}
\end{itemize}

\subsection{Path Properties}

\begin{itemize}
    \item Paths are continuous (no jumps).
    \item Nowhere differentiable.
    \item Paths are \textbf{\underline{self-similar}}, i.e.\
        \[
            (T^HB_{t_1},\ldots,T^HB_{t_n})\stackrel{d}{=}(B_{Tt_1},\ldots,B_{Tt_n})
        \]
        with $H=0.5$ is the Hurst coefficient.
\end{itemize}

\subsection{Brownian Motion with Drift}

\begin{itemize}
    \item A \textbf{\underline{Gaussian Process}} is a stochastic process s.t.
        every finite collection of those random variables has a multivariate
        normal distribution.
    \item The process is
        \[
            X_t=\mu t+\sigma B_t,\quad t\ge 0.
        \]
    \item It is a Gaussian process with the following properties:
        \begin{itemize}
            \item $\mathbb{E}[X_t]=\mu t$,
            \item Cov$(X_t, X_s)=\sigma^2\min\left\{t,s\right\}$. (Practice!)
        \end{itemize}
\end{itemize}

\subsection{Geometric Brownian Motion with Drift}

\begin{itemize}
    \item The process is
        \[
            X_t=e^{\mu t+\sigma B_t},\quad t\ge 0.
        \]
    \item It is \emph{not} Gaussian, but log-normal instead, with the following
        properties:
        \begin{itemize}
            \item $\mathbb{E}[X_t]=e^{\mu t+\frac{1}{2}\sigma^2t}$, (Practice!)
            \item
                $\text{Cov}(X_t,X_s)=e^{\left(\mu+\frac{1}{2}\sigma^2\right)(t+s)}\left(e^{\sigma^2s}-1\right)$.
                (Practice!)
        \end{itemize}
\end{itemize}


\section{Stochastic Integration}



\end{document}
