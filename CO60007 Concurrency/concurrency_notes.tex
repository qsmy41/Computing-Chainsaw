\documentclass[twocolumn,landscape,10pt]{article}
\usepackage[thinc]{esdiff} % for typesettign derivatives
\usepackage{amsthm} % provides an enhanced version of LaTex's \newtheorem command
\usepackage{mdframed} % framed environments that can split at page boundaries
\usepackage{enumitem} % bulletin points or other means of listing things
\usepackage{amssymb} % for AMS symbols
\usepackage{amsmath} % so as to use align
\usepackage{latexsym} % so as to use symbols like \leadsto
\usepackage{mathrsfs} % for using mathscr for char like operators
\usepackage{commath} % for using norm symbol
\usepackage{mathtools} % for using environments like dcases
\usepackage{authblk} % for writing affiliations
\usepackage{graphicx} % for importing images
\graphicspath{{./images/}} % for the path to images, also always put label behind captions
\usepackage{textcomp} % for using degree symbol
\usepackage{hyperref} % for clickable link in the pdf & customizable reference text
\usepackage[all]{hypcap} % for clickable link to images instead of caption
\usepackage[margin=1in]{geometry} % default is 1.5in
% \usepackage[left=0.4in, right=0.4in, top=0.8in, bottom=0.8in]{geometry}
\usepackage[title]{appendix} % for attaching appendix
\allowdisplaybreaks % allow page breaking in display maths, like align
% allow for more advanced table layout
\usepackage{booktabs}
\usepackage{multirow}
\usepackage{siunitx}
% for adjusting caption settings
\usepackage{caption}
\captionsetup[table]{skip=10pt}

\theoremstyle{definition}
\mdfdefinestyle{defEnv}{%
  hidealllines=false,
  nobreak=true,
  innertopmargin=-1ex,
}

% The following is for writing block of code
\usepackage{listings}
\usepackage{color}

\definecolor{dkgreen}{rgb}{0,0.6,0}
\definecolor{gray}{rgb}{0.5,0.5,0.5}
\definecolor{mauve}{rgb}{0.58,0,0.82}

% setting of the thickness of the 4 lines of box
\setlength{\fboxrule}{2pt}

% Use the following to change code language and related settings
\lstset{frame=tb,
  language=Python,
  aboveskip=3mm,
  belowskip=3mm,
  showstringspaces=false,
  columns=flexible,
  basicstyle={\small\ttfamily},
  numbers=none,
  numberstyle=\tiny\color{gray},
  keywordstyle=\color{blue},
  commentstyle=\color{dkgreen},
  stringstyle=\color{mauve},
  breaklines=true,
  breakatwhitespace=true,
  tabsize=3
}

\pagestyle{headings}
\author{Lectured by Azalea Raad and Alastair Donaldson}
\title{The Theory \& Practice of Concurrent Programming}
\affil{Typed by Aris Zhu Yi Qing}
\begin{document}
\maketitle

\section{Synchronisation Paradigms}

\subsection{Properties in Asynchronous computation}

\begin{enumerate}
    \item Safety
        \begin{itemize}
            \item Nothing bad happens ever
            \item If it is violated, it is done by a finite computation
        \end{itemize} 
    \item Liveness
        \begin{itemize}
            \item Something good happens eventually
            \item Cannot be violated by a finite computation
        \end{itemize} 
\end{enumerate} 

\subsection{Problems in Asynchronous computation}

\begin{enumerate}
    \item Mutual Exclusion (Safety)
        \begin{itemize}
            \item \textbf{cannot} be solved by transient communication or
                interrupts
            \item \textbf{can} be solved by shared variables that can be
                read or written
        \end{itemize} 
    \item No Deadlock (Liveness)
\end{enumerate} 

\subsection{Protocols in Asynchronous computation}

\begin{enumerate}
    \item Flag Protocol:
        \begin{itemize}
            \item Raise flag
            \item While A's flag is up
                \begin{itemize}
                    \item Lower flag
                    \item Wait for A's flag to go down
                    \item Raise flag
                \end{itemize} 
            \item Do something
            \item Lower flag
        \end{itemize} 
    \item Producer/Consumer:
        \begin{itemize}
            \item For A(producer), while flag is up wait. So when flag becomes down,
                do something, then raise the flag.
            \item For B(consumer), while flag is down, wait. So when flag
                becomes up, do something, then put down the flag.
        \end{itemize} 
    \item Readers/Writers:
        \begin{itemize}
            \item Each thread \texttt{i} has \texttt{size[i]} counter. Only it
                increments or decrements.
            \item To get object's size, a thread reads a ``snapshot'' of all
                counters.
            \item This eliminates the bottleneck of ``having exclusive access to
                the common counter''.
        \end{itemize} 
\end{enumerate} 

\subsection{Performance Measurement}

Amdahl's law:
\[
    \text{Speedup} = \frac{\text{1-thread execution time}}
    {n\text{-thread execution time}}
    = \frac{1}{1-p+\frac{p}{n}},
\]
where $p$ is the fraction of the algorithm having parallel execution, and $n$ is
the number of threads.


\end{document}
