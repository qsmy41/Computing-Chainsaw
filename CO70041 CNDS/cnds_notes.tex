\documentclass[twocolumn,landscape,10pt]{article}
\usepackage[thinc]{esdiff} % for typesettign derivatives
\usepackage{amsthm} % provides an enhanced version of LaTex's \newtheorem command
\usepackage{mdframed} % framed environments that can split at page boundaries
\usepackage{enumitem} % bulletin points or other means of listing things
\usepackage{amssymb} % for AMS symbols
\usepackage{amsmath} % so as to use align
\usepackage{latexsym} % so as to use symbols like \leadsto
\usepackage{mathrsfs} % for using mathscr for char like operators
\usepackage{commath} % for using norm symbol
\usepackage{mathtools} % for using environments like dcases
\usepackage{authblk} % for writing affiliations
\usepackage{graphicx} % for importing images
\graphicspath{{./images/}} % for the path to images, also always put label behind captions
\usepackage{textcomp} % for using degree symbol
\usepackage{hyperref} % for clickable link in the pdf & customizable reference text
\usepackage[all]{hypcap} % for clickable link to images instead of caption
\usepackage[margin=0.5in]{geometry} % default is 1.5in
% \usepackage[left=0.4in, right=0.4in, top=0.8in, bottom=0.8in]{geometry}
\usepackage[title]{appendix} % for attaching appendix
\allowdisplaybreaks % allow page breaking in display maths, like align
\usepackage{xcolor} % for setting color of a block of text, use \textcolor{<color>}{}
\usepackage[normalem]{ulem} % for strikethrough text, use \sout{}
% allow for more advanced table layout
\usepackage{booktabs}
\usepackage{multirow}
\usepackage{siunitx}
% for adjusting caption settings
\usepackage{caption}
\captionsetup[table]{skip=10pt}

\theoremstyle{definition}
\mdfdefinestyle{defEnv}{%
  hidealllines=false,
  nobreak=true,
  innertopmargin=-1ex,
}

% The following is for writing block of code
\usepackage{listings}
\usepackage{color}

\definecolor{dkgreen}{rgb}{0,0.6,0}
\definecolor{gray}{rgb}{0.5,0.5,0.5}
\definecolor{mauve}{rgb}{0.58,0,0.82}

% setting of the thickness of the 4 lines of box
\setlength{\fboxrule}{2pt}

% Use the following to change code language and related settings
\lstset{frame=tb,
  language=Python,
  aboveskip=3mm,
  belowskip=3mm,
  showstringspaces=false,
  columns=flexible,
  basicstyle={\small\ttfamily},
  numbers=none,
  numberstyle=\tiny\color{gray},
  keywordstyle=\color{blue},
  commentstyle=\color{dkgreen},
  stringstyle=\color{mauve},
  breaklines=true,
  breakatwhitespace=true,
  tabsize=3,
  literate={~} {$\sim$}{1}
}

\pagestyle{headings}
\author{Lectured by Emil C Lupu}
\title{Distributed Systems}
\affil{Typed by Aris Zhu Yi Qing}
\begin{document}
\maketitle
\tableofcontents
\newpage

\section{Characteristics}

\subsection{Distribution Transparencies}

Realize a coherent system by \emph{hiding distribution} from the user where
possible.

\begin{itemize}
    \item \textbf{Access}: uniform access whether local or remote
    \item \textbf{Location}: access without knowledge of location
    \item \textbf{Concurrency}: sharing without interference (requires synchronization)
    \item \textbf{Replication}: hides use of redundancy (e.g. for fault tolerance)
    \item \textbf{Failure}: conceal failures by replication or recovery
    \item \textbf{Migration}: hides migration of components (e.g. for load balancing)
    \item \textbf{Performance}: hide performance variations (e.g. through use of
        scheduling and reconfiguration)
    \item \textbf{Scaling}: permits expansion by adding more resources (e.g. cloud)
\end{itemize} 

\subsection{Challenges}

\begin{itemize}
    \item \textbf{Heterogeneity}: different OS, data
        representation, implementations, etc.
    \item \textbf{Openness}: need to define \emph{interfaces} for components to
        easily scale up systems
    \item \textbf{Security}: control access to preserve integrity and
        confidentiality
    \item \textbf{Concurrency}: inconsistencies may arise with interleaving
        requests
    \item \textbf{Failure handling}: transient/permanent failures could occur at
        any time. It is difficult detect them and to maintain consistency.
    \item \textbf{Scalability}: size of the system makes it difficult to
        maintain information about \emph{system state}.
\end{itemize} 

\subsection{Wrong Assumptions}

\begin{itemize}
    \item The network is reliable, secure \& homogeneous.
    \item The topology does not change.
    \item The latency is zero.
    \item The bandwidth is infinite.
    \item Transport cost is zero.
    \item There is one administrator.
\end{itemize} 

\subsection{Terminology}

\begin{itemize}
    \item \textbf{Client}: an entity initiating an interation
    \item \textbf{Server}: a componenet responds to interactions usually
        implemented as a process
    \item \textbf{Service}: a componenet of a computer system that manages a
        collection of resources and presents their functionality to users.
    \item \textbf{Middleware}: software layer between the application and the OS
        masking the heterogeneity of the underlying system.
\end{itemize} 


\section{Architecture}

\subsection{Layered architecture}
\begin{itemize}
    \item e.g. Network stack. Control flows downwards, results flow upwards.
    \item[+] framework is simple and easy to learn and implement
    \item[+] reduced dependency due to layer separation
    \item[+] testing is easier with such modularity
    \item[+] cost overheads are fairly low
    \item[-] scalability is difficult due to fixed framework structure
    \item[-] difficult to maintain, since a change in a single layer can
        affect the entire system because it operates as a single unit
    \item[-] parallel processing is not possible
\end{itemize} 

\subsection{Object-based and service-oriented architectures}
\begin{itemize}
    \item e.g. RMI
    \item[+] reusability, easy maintainability and greater reliability
        due to modularity
    \item[+] improved scalability and availability: multiple instances
        of a single service can run on different servers at the same
        time.
    \item[-] Increased overhead: Service interactions require
        validations of inputs, thereby increasing the response time and
        machine load, and reducing the overall performance
\end{itemize} 

\subsection{Message-based architectures}
\begin{table}[h]
    \centering
    \begin{tabular}{ll||c|c}
          & & Temporally coupled & Temporally decoupoled \\
        \hline\hline
        \rule{0pt}{20px} &
        \shortstack{Referentially \\coupled} & Direct process messaging & Messaging via
        mailbox \\
        \hline
        \rule{0pt}{20px} &
        \shortstack{Referentially \\decoupled} & Event-based (publish-subscribe) &
        Shared data spaces \\
    \end{tabular} 
\end{table} 
\begin{itemize}
    \item \underline{\emph{Referentially coupled}}: processes name
        sender/receiver in their communication.
    \item \underline{\emph{Temporally coupled}}: both sender and receiver need
        to be up and running.
\end{itemize} 

\subsection{Peer-to-peer}
\begin{itemize}
    \item structured: Each node is indexed so that the location is
        known, and messages are routed according to the topology.
    \item unstructured: \emph{flooding} or \emph{random walks} or both.
    \item[+] no server needed since individual workstations are used to
        access files
    \item[+] resilient to computer failures, since it does not disrupt
        any other part of the network
    \item[+] very scalable
    \item[-] poor performance with larger networks since each computer is being
        accessed by other users
    \item[-] no central file system, hard to look up or backup
    \item[-] ensuring that viruses are not introduced into the network
        is the responsibility of each individual user.
    \item[-] There is no security other than assigning permissions.
\end{itemize} 

\section{Message-passing and IPC}

\begin{itemize}
    \item \underline{Asynchronous send}: sender continues its execution once the
        message has been copied out of its address space
        \begin{itemize}
            \item[+] mostly used with \emph{blocked receive}
            \item[+] underlying system must provide buffering for receiving
                messages independently of receiver processes
            \item[+] \emph{loose} coupling: sender does not know when message will
                be received, does not suspend execution until the message has
                been received
            \item[-] \emph{Buffer exhaustion} (no flow control)
            \item[-] formal verification is more difficult, as need to account
                for the state of the buffers
        \end{itemize} 
    \item \underline{Synchronous send}: \emph{blocked send}, where the sender
        is held up until actual receipt of the message by the destination.
        \begin{itemize}
            \item[+] usually used with blocking receive, where receiver
                execution is suspended until a message is received.
            \item[+] synchronization between sender and receiver
            \item[+] generallyl easier to formally reason about synchronous
                systems
            \item[-] what if no receivers? message loss?
            \item[-] No multi-destination, requiring synchronization with all
                receivers.
            \item[-] implementation more complicated
            \item[-] The underlying communication service is expected to be
                \emph{reliable}, i.e.\ to guarantee in order message delivery.
        \end{itemize} 
    \item \underline{Asynchronous receive}: process continues execution if there
        are no messages. hardly provided as primitives
    \item \underline{Blocked receive}: the destination process blocks if no
        message is available, and receives it into a target variable when
        available.
    \item Please check the coursework for how UDP client/server is implemented in Java, 
        e.g. how datagram, socket, port are used.
\end{itemize} 



\end{document}
